\chapter{Conclusion and Recommendation}
\label{chap:Conclusion and Recommendation}

\indent \indent The Basic Genetic Algorithm could aid in optimizing solutions for problems like the TSP. Improvements such as increasing the number of offspring, such as in Multi-Offspring Genetic Algorithm, could provide better solutions compared to the BGA which only bears single child by creating a more diverse search space. However, if MO-GA is combined with another strategy like modifying the mutation rate and turning it into a dynamic one, solutions could then be more optimized by creating a balance between optimizing the population and diversifying it with the mutation operator.

\indent \indent The dynamic capabilities of the mutation rate in MO-GA with DM created a more optimized mutation capabilities without trial and error of constant mutation values, and it yielded better solutions by preventing currently fit individuals into occupying the whole population, and thus, encouraging diversification which aided into evading from getting stuck into a local optima. The solutions obtained from the MO-GA DM yielded low percentage errors from the known optimum solutions. With the significant differences in percentage errors of the MO-GA and MO-DA with DM from the known solutions, the modification made with the new algorithm is indeed an improvement.

\indent \indent Other ways of initializing parameters could be studied to create comparisons as to which greatly affects the changes on the behaviour of the population. Another interesting way to explore is tweaking the number of offspring, whether to increase it or to create an adaptive method where parents have different number of offspring. This MO-GA with Dynamic Mutation could be studied further in order to explore more possible ways of obtaining optimized solutions faster and better. Other applications for this algorithm aside from TSP could also be an interesting way to explore its solution searching capabilities. 

 
